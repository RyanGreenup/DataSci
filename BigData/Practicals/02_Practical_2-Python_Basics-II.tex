% Created 2020-07-23 Thu 19:38
% Intended LaTeX compiler: pdflatex
\documentclass[11pt]{article}
\usepackage[utf8]{inputenc}
\usepackage[T1]{fontenc}
\usepackage{graphicx}
\usepackage{grffile}
\usepackage{longtable}
\usepackage{wrapfig}
\usepackage{rotating}
\usepackage[normalem]{ulem}
\usepackage{amsmath}
\usepackage{textcomp}
\usepackage{amssymb}
\usepackage{capt-of}
\usepackage{hyperref}
\usepackage{minted}
\usepackage{/home/ryan/Dropbox/profiles/Templates/LaTeX/ScreenStyle}
\author{Ryan Greenup}
\date{\today}
\title{Big Data; 02 Practical - Python Basics 2}
\hypersetup{
 pdfauthor={Ryan Greenup},
 pdftitle={Big Data; 02 Practical - Python Basics 2},
 pdfkeywords={},
 pdfsubject={},
 pdfcreator={Emacs 26.3 (Org mode 9.4)}, 
 pdflang={English}}
\begin{document}

\maketitle
\tableofcontents

:HTML:
line-style when

\section{Calculate Data Statistics}
\label{sec:org2a6c24c}

\subsection{Sum Values}
\label{sec:org5adc63c}
So what we need to do is go through each value and tally it up, it is important
however to return the value and the print it:

\begin{minted}[]{python}
def mysum(x):
    total = float(0)
    for i in x:
        total = total + float(i)
    return total

value = mysum([1,2,3])
print(value)
\end{minted}

\begin{verbatim}
6.0
\end{verbatim}

\subsubsection{Improvements}
\label{sec:org6cf3875}

We could however improve this by using a \texttt{try} / \texttt{except} test, in the event
that a non-numerical list is provided:

\begin{minted}[]{python}
def mysum(x):
    total = float(0)
    for i in x:
        try:
            total = total + float(i)
        except:
            print("The Values of the list must be numeric")
            print("Discarding Value")
    return total

value = mysum([1,2,3, "apple"])
print(value)
\end{minted}

\begin{verbatim}
The Values of the list must be numeric
6.0
\end{verbatim}
\subsection{Minimum Value}
\label{sec:orgb8571f8}

Take the first item of the list as a candidate, for every item in the list, compare it to the candidate, if the next value is bigger that will become the new candidate, finally the candidate will be the maximum value.

Just like above we use a \texttt{try} / \texttt{except} to prevent issues.

\begin{minted}[]{python}
def mymax(list_of_vals):
    candidate = list_of_vals[0]  # Unlike R/Mathematica/Julia, python starts from 0.
    for i in list_of_vals:
        try:
            if i > candidate:
                candidate = float(i)
        except:
            print("The list items must be numeric")
            print("Discarding Value")
    return candidate

print(mymax([4, 6, 2, 5, 7, 3, 8, "john", -9]))
\end{minted}

\begin{verbatim}
The list items must be numeric
Discarding Value
8.0
\end{verbatim}

\subsection{Maximum Value}
\label{sec:orgf345f2e}

Same as above, just remember to:

\begin{itemize}
\item wrap in \texttt{float()} as appropriate
\item print the function call.
\end{itemize}


\begin{minted}[]{python}
def mymin(thelist):
    candidate = thelist[0]
    for i in thelist:
        try:
            if float(i) > float(candidate):
                candidate = i
        except:
            print("list items must be numeric, discarding value")
    return candidate

value = mymin([1, 5, 3, "apple", 8, 2, -9])
print(value)
\end{minted}

\begin{verbatim}
list items must be numeric, discarding value
8
\end{verbatim}

\section{Vector Norm, Inner Product and Distance}
\label{sec:org17d144e}
\subsection{Piping}
\label{sec:orgb056ed7}
The \emph{Toolz} Module gives something very similar to piping in bash / julia / R
\subsection{Using Built ins}
\label{sec:org693e581}
\begin{minted}[]{python}
import math as mt
import copy

## because python counts from 0 indexing is confusing,
## the count will come back as 4, but the indexes will be 0, 1, 2 and 3.

def getNorm(x):
    total = 0
    for i in range(len(x)):
        total=x[i-1]**2+total
    return mt.sqrt(total)

print(len([1,2,3,4]))


xvec = [0, 1, 2, 3, 4]
yvec = [4, 3, 2, 1, 0]

norm = getNorm(xvec);            print(norm)
\end{minted}

\begin{verbatim}
4
5.477225575051661
\end{verbatim}

\subsection{Inner Product}
\label{sec:org1a10d9c}
\begin{minted}[]{python}

def getInnerProd(x, y):
    z = copy.deepcopy(x)    ## Careful, you need to copy, not just assign
    if len(x) == len(y):
        for i in range(len(x)):
            z[i] = x[i]*y[i]
        return sum(z)
    else:
        print("The vectors must have the same dimension")
xvec = [0, 1, 2, 3, 4]
yvec = [4, 3, 2, 1, 0]

norm = getNorm(xvec);            print(norm)
norm = getNorm(yvec);            print(norm)
prod = getInnerProd(xvec, yvec); print(prod)
\end{minted}

\begin{verbatim}
5.477225575051661
5.477225575051661
10
6.324555320336759
\end{verbatim}

\subsection{Distance}
\label{sec:orgd4ca3fa}
\begin{minted}[]{python}

def getDist(x, y):
    if len(x) == len(y):
        z = mt.sqrt(getNorm(x)**2 + getNorm(y)**2 - 2 * getInnerProd(x, y))
        return z
    else:
        print("The vectors must have the same dimension")


xvec = [0, 1, 2, 3, 4]
yvec = [4, 3, 2, 1, 0]

norm = getNorm(xvec);            print(norm)
norm = getNorm(yvec);            print(norm)
prod = getInnerProd(xvec, yvec); print(prod)
dist = getDist(xvec, yvec);      print(dist)

\end{minted}

\section{Vote Counting}
\label{sec:org044ba30}

\begin{minted}[]{python}
votes = "N , Y, Y,N,n , N , N  N ,n ,y, n,N,Y, y,Y,N , N , n ,y,N"
def countVotes(ballot):
    ballot = ballot.replace(",","").replace(" ", "").upper()
    neg_ballots = ballot.count("N")
    pos_ballots = ballot.count("Y")

    # Could also have used a loop
    print(str(pos_ballots) + " Yes votes and " + str(neg_ballots) + " No votes")

countVotes(votes)
votes = ",,yyyyn,,y,y,nn,y,,nn,y"
\end{minted}

\begin{verbatim}
7 Yes votes and 13 No votes
\end{verbatim}

\section{Word Capitaliser}
\label{sec:org1ba727f}
\begin{minted}[]{python}
def capitalise(sentence):
    ## Split the words into a list
    wordsList = sentence.split()
    ## These are escape words
    EscWords = ["am", "a", "an", "the", "am", "is", "are", "and", "of", "in", "on", "with", "from", "to"]
    ## The number of words starting from 0
    for i in range(len(wordsList)-1):
        ## if not in the escape words
        if i not in EscWords:
            ## replace the ith word for a capitalized one
            wordsList[i] = wordsList[i].capitalize()
    ## Take a space and use it to join the list together
    sentence_Capitalized = " ".join(wordsList)
    ## Print the output
    print(sentence_Capitalized)
    return sentence_Capitalized

capitalise("The quick brown fox jumped over the lazy dogs")
\end{minted}

\begin{verbatim}
The Quick Brown Fox Jumped Over The Lazy dogs
\end{verbatim}

\section{Parse File}
\label{sec:org088dfe5}
\subsection{Set up the Text File}
\label{sec:org4b7a218}

Take the following text:
\begin{quote}
Unit ID, unit name, course name
301046, Big Data, MICT
300581, Programming Techniques, BICT
300144, Object Oriented Analysis, BICT
300103, Data Structure, BCS
300147, Object Oriented Programming, BCS
300569, Computer Security, BIS
301044, Data Science, MICT
300582, Technologies for Web Applications, BICT
\end{quote}

Let's write it to a file:

\begin{minted}[]{bash}
pwd
ls
\end{minted}

\begin{minted}[]{python}


scemunits = """Unit ID, unit name, course name
301046, Big Data, MICT
300581, Programming Techniques, BICT
300144, Object Oriented Analysis, BICT
300103, Data Structure, BCS
300147, Object Oriented Programming, BCS
300569, Computer Security, BIS
301044, Data Science, MICT
300582, Technologies for Web Applications, BICT"""

def writeTextFile(text, filename) :
    f = open(filename, 'w')
    for i in text.split('\n'):
        f.writelines(i+'\n')
    f.close()

writeTextFile(scemunits, 'scemunits.txt')

\end{minted}

In order to check that worked we can run \texttt{cat} from \emph{Python}:

\begin{minted}[]{python}
import subprocess
MyCommand = "cat scemunits.txt"
scemunits_txt = subprocess.run(MyCommand.split(), capture_output = True)
print(scemunits_txt)
\end{minted}

\begin{verbatim}
CompletedProcess(args=['cat', 'scemunits.txt'], returncode=0, stdout=b'Unit ID, unit name, course name\n301046, Big Data, MICT\n300581, Programming Techniques, BICT\n300144, Object Oriented Analysis, BICT\n300103, Data Structure, BCS\n300147, Object Oriented Programming, BCS\n300569, Computer Security, BIS\n301044, Data Science, MICT\n300582, Technologies for Web Applications, BICT\n', stderr=b'')
\end{verbatim}



Observe that:

\begin{enumerate}
\item The \texttt{"""} are necessary for new line strings
\item The \texttt{open(file, w)} will write over any pre-existing file (like \texttt{>} in \texttt{bash})
\begin{enumerate}
\item usint \texttt{open(file,a)} would append to a file (like \texttt{>>} in \texttt{bash})
\end{enumerate}
\item Nothing is written to disk until apter \texttt{f.close()}, that's when the changes go from memory to disk.
\end{enumerate}

\subsection{Parse the Text File}
\label{sec:orgc585340}
\subsubsection{Read the Text File}
\label{sec:orge5a1a9f}

\begin{minted}[]{python}
## Open the File
scemunits_fid = open('./scemunits.txt')

## Dispense with the first line
header = scemunits_fid.readline()

## Read the remaining Lines into a var
scemunits_txt = scemunits_fid.read()

## Print what we have
print(scemunits_txt)

## Close the file
scemunits_fid.close()
\end{minted}



\begin{verbatim}
301046, Big Data, MICT
300581, Programming Techniques, BICT
300144, Object Oriented Analysis, BICT
300103, Data Structure, BCS
300147, Object Oriented Programming, BCS
300569, Computer Security, BIS
301044, Data Science, MICT
300582, Technologies for Web Applications, BICT
\end{verbatim}

\subsubsection{Return only Matching Data}
\label{sec:org60dc81a}

\begin{minted}[]{python}
## Split each line into a list element
obs = scemunits_txt.split('\n')

## Throw away the empty line
obs = list(filter(None, obs))

## Get the Course Names
    ## Use replace so whitespace is not required after ,
courses = [ obs[i].replace(', ', ',').split(',')[2] for i in range(len(obs)) ]
units = [ obs[i].replace(', ', ',').split(',')[1] for i in range(len(obs)) ]

## Enumerate the obs so that they
obs = list(obs)

## Make an empty list for the matches
matches = []

##
for i in range(len(obs)):
    ## Don't Require whitespace after comma
    if courses[i] == "MICT":
        matches.append(obs[i])


#print([header] + join(matches).insert(header))
#print([header].append(matches))
print(matches)
matches.insert(0, header.replace('\n',''))
print("\n".join(matches))

out_fid = open('outfile.txt', "w")
# out_fid.write("\n".join(matches))
for i in matches:
    out_fid.write(i+'\n')
    print(i)

out_fid.close()
\end{minted}

\begin{verbatim}
['301046, Big Data, MICT', '301044, Data Science, MICT']
Unit ID, unit name, course name
301046, Big Data, MICT
301044, Data Science, MICT
Unit ID, unit name, course name
301046, Big Data, MICT
301044, Data Science, MICT
\end{verbatim}


We can now inspect the contents of that file:

\subsection{Wrap it into a function}
\label{sec:org869de63}

\begin{minted}[]{python}
#!/usr/bin/env python3

# * Create the Text File
scemunits = """Unit ID, unit name, course name
301046, Big Data, MICT
300581, Programming Techniques, BICT
300144, Object Oriented Analysis, BICT
300103, Data Structure, BCS
300147, Object Oriented Programming, BCS
300569, Computer Security, BIS
301044, Data Science, MICT
300582, Technologies for Web Applications, BICT"""

def writeTextFile(text, filename) :
    f = open(filename, 'w')
    for i in text.split('\n'):
        f.writelines(i+'\n')
    f.close()

writeTextFile(scemunits, 'scemunits.txt')


# * Main Functions
def readWriteFile(infile, outfile):
    readTheTextFile(infile)
    listOfLines = returnMatchingData(outfile)
    print(listOfLines)
    writeToFile(listOfLines, outfile)


# ** Sub Functions
# *** Input
def readTheTextFile(infile):
    ## Open the File
    scemunits_fid = open(infile)
    ## /////////////////// File Open ///////////////////////

    ## Dispense with the first line
    readTheTextFile.header = scemunits_fid.readline().replace('\n', '')

    ## Read the remaining lines into an attribute
    readTheTextFile.scemunits_txt = scemunits_fid.read()

    ## Close the File
    ## /////////////////// File Closed ///////////////////////
    scemunits_fid.close()


# ** Output

def returnMatchingData(outfile):
    scemunits_txt = readTheTextFile.scemunits_txt
    ## Split each line into a list element
    obs = scemunits_txt.split('\n')

    ## Throw away empty lines
    obs = list(filter(None, obs))

    ## Get the Course and Unit Names
        ## Don't Require whitespace after comma
    courses = [ obs[i].replace(', ', ',').split(',')[2] for i in range(len(obs)) ]
    units = [ obs[i].replace(', ', ',').split(',')[1] for i in range(len(obs)) ]


    ## Enumerate the obs so that they
    obs = list(obs)

    ## Make an empty list for the matches
    matches = []


    ## for each line, throw it in the list of matches if it's in the MICT course
    ##
    for i in range(len(obs)):
        if courses[i] == "MICT":
            matches.append(obs[i])

    ## Insert the header at position 1
    matches.insert(0,readTheTextFile.header)

    return matches

def writeToFile(listOfLines, outfile):
    outfile_fid = open(outfile, 'w')
    for line in listOfLines:
        print(line)
        outfile_fid.write(line+'\n')
    outfile_fid.close()

readWriteFile('scemunits.txt', 'outfile.txt')
\end{minted}

\begin{verbatim}
['Unit ID, unit name, course name', '301046, Big Data, MICT', '301044, Data Science, MICT']
Unit ID, unit name, course name
301046, Big Data, MICT
301044, Data Science, MICT
\end{verbatim}


and to confirm that it has written to the file:

\begin{minted}[]{bash}
cat outfile.txt
\end{minted}

\section{Parse Dictionary}
\label{sec:org73bad5f}

\begin{minted}[]{python}
#!/usr/bin/env python3

# * Create the Dictionary

units = {('301046', 'Big Data'): 'MICT',
         ('300581', 'Programming Techniques'): 'BICT',
         ('300144', 'OOA'): 'BICT',
         ('300103', 'Data Structures'): 'BCS',
         ('300147', 'OOP'): 'BCS',
         ('300569', 'Computer Security'): 'BIS',
         ('301044', 'Data Science'): 'MICT',
         ('300582', 'TWA'): 'BICT'}


def displayUnits(unitsDict, keyword):
    # Should Return Gracefully if the input is wrong
    # Could have used Try/Except
    if type(unitsDict) != dict:
        print("ERROR; Require Dictionary of Unit Values")
        return
    # Make an empty List to fill
    matches = []
    # For each dictionary item if it corresponds to the keyword
    # append it to the list
    for i in unitsDict:
        if units[i] == keyword:
            matches.append(i)
    # Use to get back matches[][1]
    matching_units = [matches[i][1] for i in range(len(matches))]
    # Return the Value
    return matching_units


# To Print the Values join the list together with new line characters.
# The function should return data in a list not a string
# (python => data, bash => string)

print("Match MICT \n ---------------------")
print("\n".join(displayUnits(units, 'MICT')))

print("Match BCS  \n ---------------------")
print("\n".join(displayUnits(units, 'BCS')))
\end{minted}

\begin{verbatim}
Big Data
Data Science
Data Structures
OOP
\end{verbatim}

\section{grep}
\label{sec:org95b840b}

This is easy, just loop through the lines and print if the word is in the line.

\begin{minted}[]{python}
#!/bin/python

def pygrep(filename, expr):
    try:
        inputfile_fid = open(filename)
    except:
        print("ERROR: Could note open file")
    for line in inputfile_fid:
        if expr in line:
            print(line)

pygrep("./bigdata.txt", 'Big')
pygrep("./bigdata.txt", 'technology')
\end{minted}

\begin{verbatim}
Big data is a broad term for data sets so large or complex that they are difficult to process using

their tools, and expanding capabilities make Big Data a moving target. Thus, what is considered to be "Big" in

Big data usually includes data sets with sizes beyond the ability of commonly used software tools to capture,

curate, manage, and process data within a tolerable elapsed time. Big data "size" is a constantly moving

target, as of 2012 ranging from a few dozen terabytes to many petabytes of data. Big data is a set of

definition as follows: "Big data is high volume, high velocity, and/or high variety information assets that

Big data uses inductive statistics and concepts from nonlinear system identification to infer laws

Big data can also be defined as "Big data is a large volume unstructured data which cannot be handled by

Big data can be described by the following characteristics: Volume � The quantity of data that is generated is

data under consideration and whether it can actually be considered as Big Data or not. The name �Big Data�

Variety - The next aspect of Big Data is its variety. This means that the category to which Big Data belongs

upholding the importance of the Big Data.

�complexity� of Big Data.

Big data analytics consists of 6 Cs in the integrated industry 4.0 and Cyber Physical Systems environment. 6C

implications in an article titled "Big Data Solution Offering". The methodology addresses handling big data in

Big Data Analytics for Manufacturing Applications can be based on a 5C architecture (connection, conversion,

Big Data Lake - With the changing face of business and IT sector, capturing and storage of data has emerged

Big data requires exceptional technologies to efficiently process large quantities of data within tolerable

Bus wrapped with SAP Big data parked outside IDF13.

Big data has increased the demand of information management specialists in that Software AG, Oracle

While many vendors offer off-the-shelf solutions for Big Data, experts recommend the development of in-house

The use and adoption of Big Data, within governmental processes, is beneficial and allows efficiencies in

Governmental Big Data space.

In 2012, the Obama administration announced the Big Data Research and Development Initiative, to explore how

different big data programs spread across six departments.Big data analysis played a large role in Barack

Big data analysis was, in parts, responsible for the BJP and its allies to win a highly successful Indian

benefit of big data for manufacturing. Big data provides an infrastructure for transparency in manufacturing

In order to hone into the manner in which the media utilises Big Data, it is first necessary to provide some

Practitioners in Advertising and Media approach Big Data as many actionable points of information about

The media industries process Big Data in a dual, interconnected manner:

Big Data and the IoT work in conjunction. From a media perspective, Data is the key derivative of device inter

far-reaching impacts on media efficiency. The wealth of data generated by this industry (i.e. Big Data) allows

Engineering Education. Gautam Siwach engaged at Tackling the challenges of Big Data by MIT Computer Science

In March 2012, The White House announced a national "Big Data Initiative" that consisted of six Federal

The White House Big Data Initiative also included a commitment by the Department of Energy to provide $25

The U.S. state of Massachusetts announced the Massachusetts Big Data Initiative in May 2012, which provides

Massachusetts Institute of Technology hosts the Intel Science and Technology Center for Big Data in the MIT

The European Commission is funding the 2-year-long Big Data Public Private Forum through their Seventh

presenters from various industrial companies discussed their concerns, issues and future goals in Big Data

Computational social sciences � Anyone can use Application Programming Interfaces (APIs) provided by Big Data

emergence of the typical network characteristics of Big Data". In their critique, Snijders, Matzat, and Reips

Big data has been called a "fad" in scientific research and its use was even made fun of as an absurd practice

Questions for Big Data", the authors title big data a part of mythology: "large data sets offer a higher form

often "lost in the sheer volume of numbers", and "working with Big Data is still subjective, and what it

such as pro-active reporting especially target improvements in usability of Big Data, through automated

Big data analysis is often shallow compared to analysis of smaller data sets. In many big data projects, there

Big data is a buzzword and a "vague term", but at the same time an "obsession" with entrepreneurs,

consultants, scientists and the media. Big data showcases such as Google Flu Trends failed to deliver good

election predictions solely based on Twitter were more often off than on target. Big data often poses the same

technology went public with the launch of a company called Ayasdi.

ICT4D) suggests that big data technology can make important contributions but also present unique challenges

the later utilization stage. Finally, with ubiquitous connectivity offered by cloud computing technology, the

well as queries from more than half a million third-party sellers. The core technology that keeps Amazon

form at cloud interface by providing the raw definitions and real time examples within the technology.

experiments (i.e. process a big amount of scientific data; although not with big data technology), the
\end{verbatim}

\section{Top 10 Words}
\label{sec:org874d09b}


ope

\subsection{Create the Dictionary}
\label{sec:org2014149}
So the idea here is to first try and open the file, use try/catch so that errors relating to missing files are descriptive and exit gracefully.

Then make an empty dictionary and go through each word in the list:
\begin{itemize}
\item If the words is in the list but not the dictionary
\begin{itemize}
\item Put it in the dictionary with a value of 1
\end{itemize}
\item If the word is in the dictionary then increment its value

\begin{minted}[]{python}
#!/bin/python
from operator import itemgetter

def createDict(filename):
    try:
        filename_fid = open(filename)
        filename_str = filename_fid.read()
        word_list = filename_str.lower().split(' ')
    except:
        print("ERROR: cannot open file")
        return

    matches = dict()

    for word in word_list:
        if word not in matches:
            matches[word] = 1
        else:
            matches[word] = matches[word] + 1

    return matches

createDict('scemunits.txt')
print(createDict('scemunits.txt'))
\end{minted}

\begin{verbatim}
{'unit': 2, 'id,': 1, 'name,': 1, 'course': 1, 'name\n301046,': 1, 'big': 1, 'data,': 1, 'mict\n300581,': 1, 'programming': 1, 'techniques,': 1, 'bict\n300144,': 1, 'object': 2, 'oriented': 2, 'analysis,': 1, 'bict\n300103,': 1, 'data': 2, 'structure,': 1, 'bcs\n300147,': 1, 'programming,': 1, 'bcs\n300569,': 1, 'computer': 1, 'security,': 1, 'bis\n301044,': 1, 'science,': 1, 'mict\n300582,': 1, 'technologies': 1, 'for': 1, 'web': 1, 'applications,': 1, 'bict\n': 1}
\end{verbatim}
\end{itemize}

The rest of the code was given to us and so the sorted values can be returned thusly:

\begin{minted}[]{python}
from operator import itemgetter
myDict = createDict('bigdata.txt')

sortedList = sorted(myDict.items(), key = itemgetter(1), reverse = True)

for key, value in sortedList[:10]:
    print(key, value)

\end{minted}

\begin{verbatim}
the 302
of 213
and 201
data 178
to 170
in 122
a 99
big 80
is 75
as 59
\end{verbatim}
\end{document}
