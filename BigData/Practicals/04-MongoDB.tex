% Created 2020-08-14 Fri 16:26
% Intended LaTeX compiler: pdflatex
\documentclass[11pt]{article}
\usepackage[utf8]{inputenc}
\usepackage[T1]{fontenc}
\usepackage{graphicx}
\usepackage{grffile}
\usepackage{longtable}
\usepackage{wrapfig}
\usepackage{rotating}
\usepackage[normalem]{ulem}
\usepackage{amsmath}
\usepackage{textcomp}
\usepackage{amssymb}
\usepackage{capt-of}
\usepackage{hyperref}
\usepackage{minted}
\usepackage{/home/ryan/Dropbox/profiles/Templates/LaTeX/ScreenStyle}
\author{Ryan Greenup}
\date{\today}
\title{Big Data; 02 Practical - Python Basics 2}
\hypersetup{
 pdfauthor={Ryan Greenup},
 pdftitle={Big Data; 02 Practical - Python Basics 2},
 pdfkeywords={},
 pdfsubject={},
 pdfcreator={Emacs 27.1 (Org mode 9.4)}, 
 pdflang={English}}
\begin{document}

\maketitle
\tableofcontents

:HTML:

\section{Basic MongoDB Operations}
\label{sec:orgf97b54d}
\begin{itemize}
\item \href{https://docs.mongodb.com/manual/tutorial/insert-documents/}{See the Tutorials here}
\end{itemize}
\subsection{Insert Documents}
\label{sec:orgc89a261}

\subsection{Query Documents}
\label{sec:org533f114}
\subsection{Update Documents}
\label{sec:orgacd83d9}
\subsection{Remove Documents}
\label{sec:org834a7ff}

\section{Using the \texttt{product} Database}
\label{sec:orgf92bb53}
The \url{./product.json} Database will be used here
\subsection{Using the Mongo-compass program\hfill{}\textsc{ATTACH}}
\label{sec:orgde5a1f2}
\begin{enumerate}
\item Open mongo-compass
\item Connect to the mongoDB server
\begin{enumerate}
\item Probably \url{http://localhost:38157}
\begin{enumerate}
\item If you're on SystemD check with \texttt{sudo systemctl status mongodb}
\end{enumerate}
\end{enumerate}
\end{enumerate}


\begin{center}
\includegraphics[width=.9\linewidth]{/home/ryan/Notes/Org/.attach/7f/7bd3b3-1e74-45d3-80c0-94373ead9968/_20200814_162644screenshot.png}
\end{center}



\subsection{List Movies}
\label{sec:org137624a}
\subsection{Find Songs}
\label{sec:org3edf32c}
\subsection{Calculate the Average Price of Books}
\label{sec:org5b0daf1}

\section{How to use Org-Babel Mongo}
\label{sec:orge20ee8d}
\subsection{Org Babel Mongo}
\label{sec:orgd7278cc}

Support for MongoDB queries in org-mode blocks, like so:

\begin{verbatim}
#+BEGIN_SRC mongo :db staff
db.employees.count({country: "gb"});
#+END_SRC

#+RESULTS:
: 15
\end{verbatim}

\subsubsection{Installation}
\label{sec:org51e78c9}

If you're hooked up to \href{http://melpa.milkbox.net/}{MELPA}:

\begin{verbatim}
M-x package-refresh-contents
M-x package-install RET ob-mongo
\end{verbatim}

Alternatively just grab the single \texttt{ob-mongo.el} file and install that in your preferred way.

\subsubsection{Status}
\label{sec:org4cbf32c}

Alpha. Safe to use, but feature-poor. It's still better than it not existing at all. ;-)

\subsubsection{Options}
\label{sec:org15284f9}

Each block supports the following arguments:

\begin{center}
\begin{tabular}{llll}
Argument & Description & Example & Default\\
\hline
\texttt{:db} & Database name. & \texttt{\#+BEGIN\_SRC mongo :db staff} & None.\\
\texttt{:host} & Host & \texttt{\#+BEGIN\_SRC mongo :host localhost} & None.\\
\texttt{:port} & Port & \texttt{\#+BEGIN\_SRC mongo :port 27018} & None.\\
\texttt{:user} & Username & \texttt{\#+BEGIN\_SRC mongo :user root} & None.\\
\texttt{:password} & Password & \texttt{\#+BEGIN\_SRC mongo :password superword} & None.\\
\texttt{:mongoexec} & Mongo executable & \texttt{\#+BEGIN\_SRC mongo :mongoexec mongo26} & ``mongo''\\
\end{tabular}
\end{center}

All defaults are customizable with \texttt{M-x customize-group RET ob-mongo}.
\end{document}
