% LaTeX template for academic reports (thesis)

% Sebastian Sauer



\documentclass[11pt,a4paper,oneside]{article}
\usepackage[]{lmodern}
\usepackage{amssymb,amsmath}
\usepackage{ifxetex,ifluatex}
\usepackage[nottoc]{tocbibind}



% some more packages...
\usepackage{graphicx}
\usepackage{scrpage2}
\usepackage{xcolor}
\usepackage{hyperref}
\hypersetup{colorlinks=true, linkcolor = blue, urlcolor = blue}
\usepackage{eso-pic}


\renewcommand{\baselinestretch}{1.5}  % line distance is 1.5

%\renewcommand{\chaptername}{} %% remove the word \chapter



\usepackage{fixltx2e} % provides \textsubscript
\ifnum 0\ifxetex 1\fi\ifluatex 1\fi=0 % if pdftex
  \usepackage[T1]{fontenc}
  \usepackage[utf8]{inputenc}
\else % if luatex or xelatex
  \ifxetex
    \usepackage{mathspec}
  \else
    \usepackage{fontspec}
  \fi
  \defaultfontfeatures{Ligatures=TeX,Scale=MatchLowercase}
\fi
% use upquote if available, for straight quotes in verbatim environments
\IfFileExists{upquote.sty}{\usepackage{upquote}}{}
% use microtype if available
\IfFileExists{microtype.sty}{%
\usepackage{microtype}
\UseMicrotypeSet[protrusion]{basicmath} % disable protrusion for tt fonts
}{}
\usepackage[margin=1.0in]{geometry}
\usepackage{hyperref}
\hypersetup{unicode=true,
            pdftitle={2019 Final Exam},
            pdfauthor={Ryan Greenup},
            pdfborder={0 0 0},
            breaklinks=true}
\urlstyle{same}  % don't use monospace font for urls
\usepackage{color}
\usepackage{fancyvrb}
\newcommand{\VerbBar}{|}
\newcommand{\VERB}{\Verb[commandchars=\\\{\}]}
\DefineVerbatimEnvironment{Highlighting}{Verbatim}{commandchars=\\\{\}}
% Add ',fontsize=\small' for more characters per line
\usepackage{framed}
\definecolor{shadecolor}{RGB}{248,248,248}
\newenvironment{Shaded}{\begin{snugshade}}{\end{snugshade}}
\newcommand{\AlertTok}[1]{\textcolor[rgb]{0.94,0.16,0.16}{#1}}
\newcommand{\AnnotationTok}[1]{\textcolor[rgb]{0.56,0.35,0.01}{\textbf{\textit{#1}}}}
\newcommand{\AttributeTok}[1]{\textcolor[rgb]{0.77,0.63,0.00}{#1}}
\newcommand{\BaseNTok}[1]{\textcolor[rgb]{0.00,0.00,0.81}{#1}}
\newcommand{\BuiltInTok}[1]{#1}
\newcommand{\CharTok}[1]{\textcolor[rgb]{0.31,0.60,0.02}{#1}}
\newcommand{\CommentTok}[1]{\textcolor[rgb]{0.56,0.35,0.01}{\textit{#1}}}
\newcommand{\CommentVarTok}[1]{\textcolor[rgb]{0.56,0.35,0.01}{\textbf{\textit{#1}}}}
\newcommand{\ConstantTok}[1]{\textcolor[rgb]{0.00,0.00,0.00}{#1}}
\newcommand{\ControlFlowTok}[1]{\textcolor[rgb]{0.13,0.29,0.53}{\textbf{#1}}}
\newcommand{\DataTypeTok}[1]{\textcolor[rgb]{0.13,0.29,0.53}{#1}}
\newcommand{\DecValTok}[1]{\textcolor[rgb]{0.00,0.00,0.81}{#1}}
\newcommand{\DocumentationTok}[1]{\textcolor[rgb]{0.56,0.35,0.01}{\textbf{\textit{#1}}}}
\newcommand{\ErrorTok}[1]{\textcolor[rgb]{0.64,0.00,0.00}{\textbf{#1}}}
\newcommand{\ExtensionTok}[1]{#1}
\newcommand{\FloatTok}[1]{\textcolor[rgb]{0.00,0.00,0.81}{#1}}
\newcommand{\FunctionTok}[1]{\textcolor[rgb]{0.00,0.00,0.00}{#1}}
\newcommand{\ImportTok}[1]{#1}
\newcommand{\InformationTok}[1]{\textcolor[rgb]{0.56,0.35,0.01}{\textbf{\textit{#1}}}}
\newcommand{\KeywordTok}[1]{\textcolor[rgb]{0.13,0.29,0.53}{\textbf{#1}}}
\newcommand{\NormalTok}[1]{#1}
\newcommand{\OperatorTok}[1]{\textcolor[rgb]{0.81,0.36,0.00}{\textbf{#1}}}
\newcommand{\OtherTok}[1]{\textcolor[rgb]{0.56,0.35,0.01}{#1}}
\newcommand{\PreprocessorTok}[1]{\textcolor[rgb]{0.56,0.35,0.01}{\textit{#1}}}
\newcommand{\RegionMarkerTok}[1]{#1}
\newcommand{\SpecialCharTok}[1]{\textcolor[rgb]{0.00,0.00,0.00}{#1}}
\newcommand{\SpecialStringTok}[1]{\textcolor[rgb]{0.31,0.60,0.02}{#1}}
\newcommand{\StringTok}[1]{\textcolor[rgb]{0.31,0.60,0.02}{#1}}
\newcommand{\VariableTok}[1]{\textcolor[rgb]{0.00,0.00,0.00}{#1}}
\newcommand{\VerbatimStringTok}[1]{\textcolor[rgb]{0.31,0.60,0.02}{#1}}
\newcommand{\WarningTok}[1]{\textcolor[rgb]{0.56,0.35,0.01}{\textbf{\textit{#1}}}}
\usepackage{longtable,booktabs}
\usepackage{graphicx,grffile}
\makeatletter
\def\maxwidth{\ifdim\Gin@nat@width>\linewidth\linewidth\else\Gin@nat@width\fi}
\def\maxheight{\ifdim\Gin@nat@height>\textheight\textheight\else\Gin@nat@height\fi}
\makeatother
% Scale images if necessary, so that they will not overflow the page
% margins by default, and it is still possible to overwrite the defaults
% using explicit options in \includegraphics[width, height, ...]{}
\setkeys{Gin}{width=\maxwidth,height=\maxheight,keepaspectratio}
\usepackage[normalem]{ulem}
% avoid problems with \sout in headers with hyperref:
\pdfstringdefDisableCommands{\renewcommand{\sout}{}}
\IfFileExists{parskip.sty}{%
\usepackage{parskip}
}{% else
\setlength{\parindent}{0pt}
\setlength{\parskip}{6pt plus 2pt minus 1pt}
}
\setlength{\emergencystretch}{3em}  % prevent overfull lines
\providecommand{\tightlist}{%
  \setlength{\itemsep}{0pt}\setlength{\parskip}{0pt}}
\setcounter{secnumdepth}{5}
% Redefines (sub)paragraphs to behave more like sections
\ifx\paragraph\undefined\else
\let\oldparagraph\paragraph
\renewcommand{\paragraph}[1]{\oldparagraph{#1}\mbox{}}
\fi
\ifx\subparagraph\undefined\else
\let\oldsubparagraph\subparagraph
\renewcommand{\subparagraph}[1]{\oldsubparagraph{#1}\mbox{}}
\fi


\usepackage{csquotes}

\usepackage{url}
\def\UrlBreaks{\do\/\do-}




\makeatletter
\newenvironment{kframe}{%
\medskip{}
\setlength{\fboxsep}{.8em}
 \def\at@end@of@kframe{}%
 \ifinner\ifhmode%
  \def\at@end@of@kframe{\end{minipage}}%
  \begin{minipage}{\columnwidth}%
 \fi\fi%
 \def\FrameCommand##1{\hskip\@totalleftmargin \hskip-\fboxsep
 \colorbox{shadecolor}{##1}\hskip-\fboxsep
     % There is no \\@totalrightmargin, so:
     \hskip-\linewidth \hskip-\@totalleftmargin \hskip\columnwidth}%
 \MakeFramed {\advance\hsize-\width
   \@totalleftmargin\z@ \linewidth\hsize
   \@setminipage}}%
 {\par\unskip\endMakeFramed%
 \at@end@of@kframe}
\makeatother

\makeatletter
\@ifundefined{Shaded}{
}{\renewenvironment{Shaded}{\begin{kframe}}{\end{kframe}}}
\makeatother







%
% %   \usepackage{framed}
%   \usepackage{xcolor}
%   \let\oldquote=\quote
%   \let\endoldquote=\endquote
%   \colorlet{shadecolor}{orange!15}
%   \renewenvironment{quote}{\begin{shaded*}\begin{oldquote}}{\end{oldquote}\end{shaded*}}
%   % thanks to @JLDiaz at this post: https://tex.stackexchange.com/questions/179982/add-a-black-border-to-block-quotations
% 

\newcommand{\ts}{\thinspace}



\input{/home/ryan/Dropbox/profiles/Template/LaTeX/texnotePreamble.sty}





\begin{document}  %%%%%%% main document %%%%%%%%%%%%%%%

\pagenumbering{Roman}

\begin{titlepage}

    \begin{center}
           \begin{figure}
			\centering
			%\vspace{1.5cm}
			\includegraphics[width=3cm]{/home/ryan/Dropbox/Studies/IntrotoDataSci/ExamProject/logo.png}
		\end{figure}

       \vspace*{1cm}

       \LARGE{ \textbf{   \uppercase{2019 Final Exam}   }}

       %\vspace{0.5cm}

       \large{   \uppercase{Submission Paper for Final Examination}   }

        \vspace{2cm}

         Introduction to Data Science \\ 
        %\vspace{0.5cm}
         \large{BSc(Math)}  \\ 

        \vspace{1cm}

         \textbf{Ryan Greenup}  \\ 
         Student ID: 1780-5315 \\ 
         Parramatta South  \\ 

         \vspace{1cm}
         24. 10. 2019 \\ 

        
         \vfill



     \end{center}
    \thispagestyle{empty}
\end{titlepage}

\newpage
% \thispagestyle{empty}
% \mbox{}


 \begin{abstract}
     Yart provides an RMarkdown template for rendering TeX based PDFs. It
     provides a format suitable for academic settings. The typical RMarkdown
     variables may be used. In additiion, some variabels useful for academic
     reports have been added such as name of referee, due date, course title,
     field of study, addres of author, and logo, and a few more maybe. In
     addition, paper format (eg., paper size, margins) may be adjusted; the
     babel language set of Latex is supported. Those variables are defined in
     the yaml header of the yart document. Adjust those variables to your
     need. Note that citations, figure/ table referencing is possible due to
     the underlying pandoc magic. This template is not much more than setting
     some of the variables provided by rmarkdown (pandoc, knitr, latex, and
     more), credit is due to the original authors. Please reade the rmarkdown
     documentation for detailled information on how to use rmarkdown and how
     to change settings.
 \end{abstract}
 \newpage




\newpage
\listoftables
\newpage
\listoffigures

{
\setcounter{tocdepth}{3}
\tableofcontents
}

\newpage
\pagenumbering{arabic}
\hypertarget{my-section-header-1}{%
\section{My Section Header 1}\label{my-section-header-1}}

\begin{Shaded}
\begin{Highlighting}[]
\KeywordTok{library}\NormalTok{(tidyverse)}
\end{Highlighting}
\end{Shaded}

\begin{verbatim}
#  -- Attaching packages ------------------------------------------------------------------------------------------------------------- tidyverse 1.2.1.9000 --
\end{verbatim}

\begin{verbatim}
#  v ggplot2 3.2.1           v purrr   0.3.2      
#  v tibble  2.1.3           v dplyr   0.8.3      
#  v tidyr   0.8.99.9000     v stringr 1.4.0      
#  v readr   1.3.1           v forcats 0.4.0
\end{verbatim}

\begin{verbatim}
#  -- Conflicts --------------------------------------------------------------------------------------------------------------------- tidyverse_conflicts() --
#  x dplyr::filter() masks stats::filter()
#  x dplyr::lag()    masks stats::lag()
\end{verbatim}

\begin{Shaded}
\begin{Highlighting}[]
\NormalTok{mpg}\OperatorTok{$}\NormalTok{cyl <-}\StringTok{ }\KeywordTok{factor}\NormalTok{(mpg}\OperatorTok{$}\NormalTok{cyl)}
\KeywordTok{ggplot}\NormalTok{(mpg, }\KeywordTok{aes}\NormalTok{(}\DataTypeTok{x =}\NormalTok{ cty, }\DataTypeTok{y =}\NormalTok{ hwy, }\DataTypeTok{col =}\NormalTok{ cyl)) }\OperatorTok{+}
\StringTok{  }\KeywordTok{geom_point}\NormalTok{() }\OperatorTok{+}
\StringTok{  }\KeywordTok{theme_classic}\NormalTok{()}
\end{Highlighting}
\end{Shaded}

\includegraphics{LatexTest_files/figure-latex/unnamed-chunk-1-1.pdf}

Please see the documentation of
\href{http://rmarkdown.rstudio.com/}{RMarkdown} for more details on how
to write RMarkdown documents.

Download a testlogo from here:
\url{https://raw.githubusercontent.com/sebastiansauer/yart/master/docs/logo.png}
and uncomment the respective line in the header.

For finetuning of design options, please check the tex template. There
you will find some variables such as \texttt{\$classoption\$}. Those
variables may be addressed in the yaml header of the yart file.

\hypertarget{my-section-header-2}{%
\subsection{My Section Header 2}\label{my-section-header-2}}

\enquote{Lorem ipsum} dolor sit amet, consectetur adipiscing elit. Proin
mollis dolor vitae tristique eleifend. Quisque non ipsum sit amet velit
malesuada consectetur. Praesent vel facilisis leo. Sed facilisis varius
orci, ut aliquam lorem malesuada in. Morbi nec purus at nisi fringilla
varius non ut dui. Pellentesque bibendum sapien velit. Nulla purus
justo, congue eget enim a, elementum sollicitudin eros. Cras porta augue
ligula, vel adipiscing odio ullamcorper eu. In tincidunt nisi sit amet
tincidunt tincidunt. Maecenas elementum neque eget dolor
\href{http://example.com}{egestas fringilla}:

\begin{quote}
Nullam eget dapibus quam, sit amet sagittis magna. Nam tincidunt, orci
ac imperdiet ultricies, neque metus ultrices quam, id gravida augue
lacus ac leo.
\end{quote}

Vestibulum id sodales lectus, sed scelerisque quam. Nullam auctor mi et
feugiat commodo. Duis interdum imperdiet nulla, vitae bibendum eros
placerat non. Cras ornare, risus in faucibus malesuada, libero sem
fringilla quam, ut luctus enim sapien eget dolor.

\begin{itemize}
\item
  Aufzählungen (nummeriert oder nicht) sind möglich.
\item
  Sonderzeichen werden unterstützt: äüß.
\item
  \LaTeX wird unterstützt.
\item
  Und damit auch \enquote{schöne} Formeln: \(e^{ln(e)}=e\) (stimmt
  das?).
\item
  Ein Überblick zur \textbf{Markdown-Syntax} findet sich
  \href{http://pandoc.org/README.html\#pandocs-markdown}{hier}.
\item
  Ein paar Gimmicks: H\textsubscript{2}O, This \sout{is deleted text.},
  feas\emph{ible}, not feas\emph{able}, lang---ganz lang.
\item
  Use \texttt{\textbackslash{}ts} as a shorthand for
  \texttt{\textbackslash{}thinspace} to get \enquote{z.\ts B.} instead
  of \enquote{z. B.} (thin space between the two letters)
\item
  Footnotes are supported\footnote{Fußnoten sind bei Pandoc eine Art von
    Links.}.
\item
  Zitationen sind möglich, im beliebigen Format, z.B. APA6. Das Format
  wird über die Variable \texttt{cls} definiert (im Kopfteil oben). Die
  entsprechende Datei muss im gleichen Ordner liegen wie diese
  Rmd-Datei. Die Datei mit den bibliographischen Informationen wird über
  die Variable \texttt{bibliography} angegeben. Auch diese Datei muss
  sich im gleichen Ordner befinden wie diese Rmd-Datei.
\item
  Besonders schön ist es, dass man \href{https://cran.r-project.org}{R}
  direkt einbinden kann über \href{http://yihui.name/knitr/}{knitr}.
  \href{http://galahad.well.ox.ac.uk/repro/}{Hier} findet sich eine gute
  Anleitung.
\end{itemize}

We report how we determined our sample size, all data exclusions (if
any), all manipulations, and all measures in the study.

\hypertarget{r-code}{%
\section{R-Code}\label{r-code}}

So bindet man R-Code ein:

\begin{Shaded}
\begin{Highlighting}[]
\NormalTok{x <-}\StringTok{ }\KeywordTok{c}\NormalTok{(}\DecValTok{1}\NormalTok{,}\DecValTok{2}\NormalTok{,}\DecValTok{3}\NormalTok{)}
\KeywordTok{mean}\NormalTok{(x)}
\end{Highlighting}
\end{Shaded}

\begin{verbatim}
#  [1] 2
\end{verbatim}

\hypertarget{citation}{%
\section{Citation}\label{citation}}

Put the file with the references in the same folder as the rmd-file.
Uncomment/insert a line in the yaml header such as
\texttt{bibliography:\ bib.bib}, where \texttt{bib.bib} is the name of
your bib-file. Similarly, if you want to format the citation in a
certain style, put the respective csl-file in the same folder as this
document and uncomment/insert this line in the yaml header:
\texttt{csl:\ apa6.csl}, where \texttt{apa6.csl} is the style file.

Use this format for citation: \texttt{{[}\textbackslash{}@bibtexkey{]}}.
Put all the bibliography data in one bibliography file.

Don't forget to cite software and data. R and R packages can be cited in
the following way:

\begin{Shaded}
\begin{Highlighting}[]
\KeywordTok{citation}\NormalTok{()}
\KeywordTok{citation}\NormalTok{(}\StringTok{"rmarkdown"}\NormalTok{)}
\end{Highlighting}
\end{Shaded}

Lorem ipsum dolor sit amet, consectetur adipiscing elit. Proin mollis
dolor vitae tristique eleifend. Quisque non ipsum sit amet velit
malesuada consectetur.

\hypertarget{tabellen}{%
\section{Tabellen}\label{tabellen}}

Lorem ipsum dolor sit amet, consectetuer adipiscing elit. Aenean commodo
ligula eget dolor. Aenean massa. Cum sociis natoque penatibus et magnis
dis parturient montes, nascetur ridiculus mus. Donec quam felis,
ultricies nec, pellentesque eu, pretium quis, sem. Nulla consequat massa
quis enim. Donec pede justo, fringilla vel, aliquet nec, vulputate eget,
arcu. In enim justo, rhoncus ut, imperdiet a, venenatis vitae, justo.
Nullam dictum felis eu pede mollis pretium. Integer tincidunt. Cras
dapibus. Vivamus elementum semper nisi. Aenean vulputate eleifend
tellus. Aenean leo ligula, porttitor eu, consequat vitae, eleifend ac,
enim. Aliquam lorem ante, dapibus in, viverra quis, feugiat a, tellus.
Phasellus viverra nulla ut metus varius laoreet.

So erstellt man \enquote{von Hand} eine Tabelle in Markdown:

\begin{center}\rule{0.5\linewidth}{\linethickness}\end{center}

\begin{verbatim}
 Right    Left     Center     Default
-------    ------ ----------   -------
     12     12        12            12
    123     123       123          123
      1     1          1             1
      
Table: Table caption
\end{verbatim}

\begin{center}\rule{0.5\linewidth}{\linethickness}\end{center}

Das ist das Ergebnis:

\begin{longtable}[]{@{}rlcl@{}}
\caption{Table caption}\tabularnewline
\toprule
Right & Left & Center & Default\tabularnewline
\midrule
\endfirsthead
\toprule
Right & Left & Center & Default\tabularnewline
\midrule
\endhead
12 & 12 & 12 & 12\tabularnewline
123 & 123 & 123 & 123\tabularnewline
1 & 1 & 1 & 1\tabularnewline
\bottomrule
\end{longtable}

There are comfortable and powerful R packages available for rendering
markdown tables such as Huxtable, or xtable, and other.

Table with R package \texttt{xtable}; note that this package needs to be
installed to run this example.

\begin{Shaded}
\begin{Highlighting}[]
\KeywordTok{data}\NormalTok{(mtcars)}

\KeywordTok{library}\NormalTok{(xtable)}
\KeywordTok{print.xtable}\NormalTok{(}
  \KeywordTok{xtable}\NormalTok{(}\KeywordTok{head}\NormalTok{(daten), }
         \DataTypeTok{label=}\StringTok{"tab:daten"}\NormalTok{, }
         \DataTypeTok{caption=}\StringTok{"Datenstruktur für eine within-Analyse"}\NormalTok{), }
  \DataTypeTok{comment=}\OtherTok{FALSE}\NormalTok{)}
\end{Highlighting}
\end{Shaded}

\hypertarget{figures}{%
\section{Figures}\label{figures}}

Use knit to insert images. Figures can be referenced, too.

\begin{Shaded}
\begin{Highlighting}[]
\NormalTok{knitr}\OperatorTok{::}\KeywordTok{include_graphics}\NormalTok{(}\StringTok{"/docs/picture2.png"}\NormalTok{)}
\end{Highlighting}
\end{Shaded}

\hypertarget{references}{%
\section{References}\label{references}}

{[}If some literature is cited, it appears here{]}

\setlength{\parindent}{-0.5in}
\setlength{\leftskip}{0.5in}

\end{document}
