\documentclass[12pt]{article}


%%%%%%%%%%%%%%%%%%%%%%%%%%%%%%
%%%%%%%%Preamble%%%%%%%%%%%%%%
%%%%%%%%%%%%%%%%%%%%%%%%%%%%%%

%Packages ------------------------------------

\usepackage{amsmath, amssymb, amsthm}
\usepackage[utf8]{inputenc}
%\usepackage[framemethod=tikz]{mdframed}
\usepackage{pgfplots}
\usepackage{tikz, pstricks}
\usepackage{graphicx}
%\usetikzlibrary{calc}
%\usepackage{chngcntr}
\usepackage[T1]{fontenc}
\usepackage{extsizes} % More Font Sizes
\usepackage[utf8]{inputenc}
\usepackage{amsmath}
\usepackage{sectsty} %Need it for underlining Sections
\usepackage{listings}
\usepackage{lmodern}
\usepackage{amssymb,amsmath}
\usepackage{enumitem}
\usepackage{ifxetex,ifluatex}
\usepackage{lipsum}
\usepackage[T1]{fontenc}
\usepackage[utf8]{inputenc}
\usepackage[margin=1in]{geometry}
\addtolength{\skip\footins}{2pc plus 5pt} %Add whitespace before footnotes`
%\usepackage[svgnames, x11names, dvipsnames]{xcolor} %This is included with mdframed
%\usepackage{tgadventor}
%\usepackage{titlesec} % Section Colours




%Hyperlinks-----------------------------------
\usepackage{hyperref}
\hypersetup{
	colorlinks,
	citecolor=black,
	filecolor=black,
	linkcolor=black,
	urlcolor=black,
	 colorlinks=true, %set true if you want colored links
	linktoc=all     %set to all if you want both sections and subsections linked
}

%Remove Section Numbers----------------
\makeatletter
\renewcommand{\@seccntformat}[1]{}
\makeatother

%Change the Font------------------------------
%\usepackage{PoiretOne}
\renewcommand*\familydefault{\sfdefault} %% Only if the base font of the document is to be sans serif

% Add rule after sections ------------------------------
\usepackage{titlesec}


%Listings----------------------------------------


\definecolor{dkgreen}{rgb}{0,0.6,0}
\definecolor{gray}{rgb}{0.5,0.5,0.5}
\definecolor{mauve}{rgb}{0.58,0,0.82}
\lstset{
%  frame=tb,
  frame=leftline,
  framesep=15pt,
  language=Java,
  aboveskip=15pt,
  belowskip=20pt,
  showstringspaces=false,
  columns=flexible,
  basicstyle={\small\ttfamily},
  numbers=none,
%  backgroundcolor=\color{Snow2},
  numberstyle=\tiny\color{gray},
  keywordstyle=\color{blue},
  commentstyle=\color{dkgreen},
  stringstyle=\color{mauve},
  breaklines=true,
  breakatwhitespace=true,
  tabsize=3,
  xleftmargin=1in,
}

	% Formatting----------------------------------------
	\widowpenalty=1000
	\clubpenalty=1000

% Problem boxes --------------------------------------

%\newtheorem*{prob}{Problem}
%\theoremstyle{working}{\cmss}
%\newtheorem*{working}{Worked Solution}
\renewcommand{\qedsymbol}{$\blacksquare$}


\newenvironment{prob}[1][Problem]{%
	\sffamily \itshape   %
}{\endproof} %\itshape for italics

\newenvironment{sol}[1][Problem]{%
	\proof[\nopunct]  %
}{\endproof}

%\renewenvironment{proof}{{\bfseries \fontfamily{ccr} \selectfont Proof}}{*something*}



%%%%%%%%%%%%%%%%%%%%%%%%%%%%%%%%%%%%%%%%%%%%%
%%%%%%Heading Colurs and Format%%%%%%%%%%%%%%
%%%%%%%%%%%%%%%%%%%%%%%%%%%%%%%%%%%%%%%%%%%%%
% Numbering
\renewcommand{\thesection}{}
\renewcommand{\thesubsection}{}


%%%% Define Heading colours
\definecolor{colsse}{RGB}{136, 73, 143}
\definecolor{colsss}{RGB}{156, 118, 160}
\definecolor{colspg}{RGB}{218, 168, 224}
\definecolor{coltit}{RGB}{84, 65, 86}
\definecolor{colname}{RGB}{236, 66 ,255}
		
	
	
% Change Section Colour
\titleformat{\section}
{\color{colsse}\normalfont\LARGE\bfseries}
{\color{colsse}\thesection}{-1em}{}
\fontfamily{cmss}\selectfont

%SubSection
\titleformat{\subsection}
{\color{colsss}\normalfont\Large\bfseries}
{\color{colsss}\thesubsection}{-1em}{}[\rule{4 in}{2pt}]
\fontfamily{cmss}\selectfont
% Change Font
\fontfamily{cmss}\selectfont



%Paragraph
\titleformat{\paragraph}
{\color{colsss}\normalfont\large}
{\color{colsss}\theparagraph}{9em}{}[\rule{2 in}{1 pt}]
\fontfamily{cmss}\selectfont
% Change Font
\fontfamily{cmss}\selectfont


%%%%%%%Title
\title{\color{coltit} \Huge Introduction to Data Science Final}
\author{Ryan Greenup ; 1780-5315}





%%%%%%%%%%%%%%%%%%%%%%%%%%%%%%
%%%%%%Document%%%%%%%%%%%%%%%%
%%%%%%%%%%%%%%%%%%%%%%% %%%%%%%

\maketitle
\tableofcontents

% Options for packages loaded elsewhere
\PassOptionsToPackage{unicode}{hyperref}
\PassOptionsToPackage{hyphens}{url}
%
\documentclass[
]{article}
\usepackage{lmodern}
\usepackage{amssymb,amsmath}
\usepackage{ifxetex,ifluatex}
\ifnum 0\ifxetex 1\fi\ifluatex 1\fi=0 % if pdftex
  \usepackage[T1]{fontenc}
  \usepackage[utf8]{inputenc}
  \usepackage{textcomp} % provide euro and other symbols
\else % if luatex or xetex
  \usepackage{unicode-math}
  \defaultfontfeatures{Scale=MatchLowercase}
  \defaultfontfeatures[\rmfamily]{Ligatures=TeX,Scale=1}
\fi
% Use upquote if available, for straight quotes in verbatim environments
\IfFileExists{upquote.sty}{\usepackage{upquote}}{}
\IfFileExists{microtype.sty}{% use microtype if available
  \usepackage[]{microtype}
  \UseMicrotypeSet[protrusion]{basicmath} % disable protrusion for tt fonts
}{}
\makeatletter
\@ifundefined{KOMAClassName}{% if non-KOMA class
  \IfFileExists{parskip.sty}{%
    \usepackage{parskip}
  }{% else
    \setlength{\parindent}{0pt}
    \setlength{\parskip}{6pt plus 2pt minus 1pt}}
}{% if KOMA class
  \KOMAoptions{parskip=half}}
\makeatother
\usepackage{fancyvrb}
\usepackage{xcolor}
\IfFileExists{xurl.sty}{\usepackage{xurl}}{} % add URL line breaks if available
\IfFileExists{bookmark.sty}{\usepackage{bookmark}}{\usepackage{hyperref}}
\hypersetup{
  pdftitle={06\_Practical; Trees},
  pdfauthor={Ryan Greenup 17805315},
  hidelinks,
  pdfcreator={LaTeX via pandoc}}
\urlstyle{same} % disable monospaced font for URLs
\VerbatimFootnotes % allow verbatim text in footnotes
\usepackage{listings}
\newcommand{\passthrough}[1]{#1}
\lstset{defaultdialect=[5.3]Lua}
\lstset{defaultdialect=[x86masm]Assembler}
\setlength{\emergencystretch}{3em} % prevent overfull lines
\providecommand{\tightlist}{%
  \setlength{\itemsep}{0pt}\setlength{\parskip}{0pt}}
\setcounter{secnumdepth}{-\maxdimen} % remove section numbering

\title{06\_Practical; Trees}
\author{Ryan Greenup 17805315}
\date{23 August 2019}

\begin{document}
\maketitle

```\{r setup, include=FALSE, include = FALSE, results = ``hide'', eval =
TRUE\} knitr::opts\_chunk\$set(echo = TRUE)

if(require(`pacman'))\{ library(`pacman') \}else\{
install.packages(`pacman') library(`pacman') \}

pacman::p\_load(caret, scales, ggplot2, rmarkdown, shiny, ISLR, class,
BiocManager, corrplot, plotly, tidyverse, latex2exp, stringr, reshape2,
cowplot, ggpubr, rstudioapi, wesanderson, RColorBrewer, colorspace,
gridExtra, grid, car, boot, colourpicker, tree, ggtree, mise, rpart,
rpart.plot, knitr, MASS, magrittr) \#Mass isn't available for R
3.5\ldots{}

set.seed(1) \# Set the seed such that we might have comparable results

\hypertarget{wipe-all-the-damn-variables-to-prevent-mistakes}{%
\section{Wipe all the damn variables to prevent
mistakes}\label{wipe-all-the-damn-variables-to-prevent-mistakes}}

mise()

kabstr \textless- function(df)\{

data.frame(variable = names(df), class = sapply(df, typeof),
first\_values = sapply(df, function(x) paste0(head(x), collapse =
``,'')), row.names = NULL) \%\textgreater\% kable()

\} \# Pretty Structure Print

\begin{lstlisting}

# (Wk 6) Tree Based Methods
Material of Tue 30 August 2019, week 6

## Classification Trees

Let's see what variables are predictive of high car sales, the number of sales is continuous, in order to fit a classification tree we will construct a binary variable.

When using `tree` the response variable MUST be a factor, be mindful of that.

When transforming columns to factors:

* Don't use `as.factor` because it doesn't allow specifying order
* Use tibbles because theyll tell you whether or not the column is ordered
* feed the column into `factor()`, specify the `levels = c()` in ascending order, specify `ordered = TRUE`


### Preview the data:

```{r}
CSeat.tb <- as_tibble(Carseats)


thresh <- (mean(CSeat.tb$Sales)+0.5*sd(CSeat.tb$Sales)) %>% round()

CSeat.tb$CatSales <- ifelse(CSeat.tb$Sales > thresh, "High", "Low")
CSeat.tb$CatSales <- factor(x = CSeat.tb$CatSales, levels = c("Low", "High"), ordered = TRUE)




CSeat.tb <- CSeat.tb[,c(12, 2:11)]
    
  # Anotherway to reorder
    dplyr::select(CSeat.tb, CatSales, everything())
    
    
\end{lstlisting}

\hypertarget{fit}{%
\subsubsection{Fit}\label{fit}}

Now let's use all of the data to create a classification tree of the
sales category.

Remember that we can use \passthrough{\lstinline!.!} to represent all
the variables of a data frame;

\begin{itemize}
\tightlist
\item
  If \passthrough{\lstinline!Sales!} was still in the data frame we
  could pull it out by prefixing it with \passthrough{\lstinline!-!} but
  I think that's awfully confusing, instead I would recommend specifying
  a subset that pulls out undesired columns or just making another
  dataframe, it's way too easy to make a mistake like that IMO
\end{itemize}

\begin{lstlisting}
CatSalesModTree <- tree(formula = CatSales ~ ., data = as.data.frame(CSeat.tb))
CatSalesModTree
nrow(CSeat.tb)

CatSalesModTree %>% summary()
\end{lstlisting}

\hypertarget{plot-the-model}{%
\subsubsection{Plot the Model}\label{plot-the-model}}

\hypertarget{base}{%
\paragraph{Base}\label{base}}

This summary is next to useless so let's plot it, when plotting it it is
necessary to call \passthrough{\lstinline!text!} afterwards in order
order to have lables:

\begin{lstlisting}
plot(CatSalesModTree)
text (CatSalesModTree, pretty = 0, cex = 0.6)

# #
# CatSalesModTree.rpart <- rpart(formula = CatSales ~ ., data = CSeat.tb)
# rpart.plot(CatSalesModTree.rpart, box.palette="RdBu", shadow.col="gray", nn=TRUE)
\end{lstlisting}

\hypertarget{use-rpart}{%
\paragraph{\texorpdfstring{Use
\texttt{rpart}}{Use rpart}}\label{use-rpart}}

The base plot looks truly awful, instead I would recommend using the
\passthrough{\lstinline!rpart!} package, it creates a tree model using
identical syntax to \passthrough{\lstinline!tree!} and then you can call
\passthrough{\lstinline!rpart.plot!} directly over the model and it will
work.

Be mindful that the \passthrough{\lstinline!tree!} package will split
the data until each node has less than 20 observations, whereas the
`rpart' package automatically performs 10-fold cross validation
\footnote{\href{https://www.rdocumentation.org/packages/rpart/versions/4.1-15/topics/rpart.control}{rpart.control}
  has a default list entry of \passthrough{\lstinline!xval=10!} which
  corresponds to the number of CV folds.}, the ratio contained in the
plot is the ratio of terms satisfying the specified condition and the
percentage is the percentage of all observations thence contained
\footnote{Refer to the
  \href{http://www.milbo.org/rpart-plot/prp.pdf}{rpart Vignette}}

\begin{lstlisting}
#plot(CatSalesModTree)
#text (CatSalesModTree, pretty = 0, cex = 0.6)

CatSalesModTree.rpart <- rpart(formula = CatSales ~ ., data = CSeat.tb)
rpart.plot(CatSalesModTree.rpart, box.palette="OrGy", shadow.col="gray", nn=TRUE)
\end{lstlisting}

I have no clue if an \passthrough{\lstinline!rpart!} tree object behaves
well so I'm just going to use one for plotting and the other for
modelling up until I have time to look into them.

\hypertarget{evaluate-the-tree-by-using-a-validation-split}{%
\subsubsection{Evaluate the tree by using a validation
Split}\label{evaluate-the-tree-by-using-a-validation-split}}

In order to very quickly evaluate the performance of this model, split
the data and use the model on unseen data.

\begin{lstlisting}
train <- sample(1:nrow(Carseats), 200)

#Create the Categorical variable

CSeat <-  Carseats
CSeat$CatSales <- ifelse(Carseats$Sales < round(mean(Carseats$Sales)), "Low", "High")
CSeat$CatSales <- factor(CSeat$CatSales, levels = c("Low", "High"), ordered = TRUE)

#Create the model on the training data only
carseats.tree <- tree(CatSales ~ ., data = CSeat[train,names(CSeat) != "Sales"])
 # plot(carseats.tree) ;text(carseats.tree, pretty = 0, cex = 0.6)

carseats.tree.rpart <- rpart(CatSales ~ ., data = CSeat[train,-1], model = TRUE) # Model = True stops later errors
#rpart.plot(carseats.tree.rpart, nn = TRUE)

# Create Predictions on the testing data
test.pred <- predict(object = carseats.tree, newdata = CSeat[-train,], type = "class" )

# Filter out the observations in the testing data
test.obs <- CSeat$CatSales[-train]

# Now create the confusion Matrix
  # This package prevents making mistakes
  conf.mat <-   caret::confusionMatrix(data = test.pred, reference = test.obs)
  # This could otherwise be created by using, always go prediction, reference as a standard
  table("prediction" = test.pred, "reference" = test.obs)
  
\end{lstlisting}

The Misclassification Rate for unseed testing data is
\passthrough{\lstinline!r   (1-conf.mat$overall[1] )\%>\% round(2) \%>\% as.numeric() \%>\% percent()!}
which is reasonably low, suggesting that the accuracy of this model is
acceptable, particularly because this models primary purpose is
interetability not predictive capacity.

\hypertarget{tree-pruning}{%
\subsubsection{Tree Pruning}\label{tree-pruning}}

This model may be overfit, it is necessary instead to determine whether
or not to `prune' the tree, that is reduce the model flexibility by
reducing the number of nodes at the end of the tree. So, in this case,
the number of nodes at the end of the tree is an indicator of the model
flexibility, by default, the \passthrough{\lstinline!tree!} package
creates nodes by choosing a threshold that minimises the RSS until the
number of observations in each node is 20.

The problem with this strategy is that the model may be overfit, it may
be to flexible and the results will suffer from very little model bias
but far too much variance.

The process of reducing flexibility in order to improve the model by
balancing bias and variation is called `pruning'.

In order to prune the tree the best strategy to implement is 10-fold
cross validation, fourtunately this is all built in and we don't really
have to think about it, the \passthrough{\lstinline!cv.tree()!} function
will do it automatically.

\hypertarget{using-the-the-cv.tree-package}{%
\paragraph{\texorpdfstring{Using the the \texttt{cv.tree()}
package}{Using the the cv.tree() package}}\label{using-the-the-cv.tree-package}}

The \passthrough{\lstinline!cv.tree!} package minimises the deviance by
default, the texbook provides \footnote{classref} that the
misclassification rate is preferable if preiction accuracy of the final
purened tree is the goal, deviance is an analogy to RSS in the
classification sence \footnote{\href{https://stats.stackexchange.com/a/6610}{What
  is Deviance}}, but this is outside the scope of this unit.

In order to specify to \passthrough{\lstinline!cv.tree!} that the cross
validation process should aim to minimise the `\textbf{misclassification
rate}' rather than the `\textbf{deviance}' specify the function as
\passthrough{\lstinline!prune.misclass!} to
\passthrough{\lstinline!cv.tree()!}; be super careful, because we are
changing the function, the \passthrough{\lstinline!dev!} output in the
list will NOT be deviance, the values will be misclassificaiton

\begin{lstlisting}
set.seed(3)
 tree.mod.cv <- cv.tree(object = carseats.tree, FUN = prune.misclass)
# tree.mod.cv <- cv.tree(object = CatSalesModTree, FUN = prune.misclass)
names(tree.mod.cv)[2] <- "misclassification"

tree.mod.cv #%>% str()

\end{lstlisting}

This gives us a correspondence between the limiting size of the final
node (which is inversely proportional to the flexibility of the model)
and the expected misclassification rate on testing data.

Because this is Cross Validation, the error will be an estimation of
error unseen by the model, hence the size that reflects the minimum
value should be returned.

\hypertarget{plot-the-testing-error}{%
\paragraph{Plot the Testing Error}\label{plot-the-testing-error}}

\begin{lstlisting}
nodeSize <- factor(tree.mod.cv$size, ordered = TRUE)
cv.error <- tibble(nodeSize = tree.mod.cv$size, error = tree.mod.cv$misclassification)

minval <- cv.error[cv.error$error<=min(cv.error$error),]
minNode <- min(minval$nodeSize)

ggplot(cv.error, aes(x = nodeSize, y = error), groups = 1) +
  geom_point(size = 4, col = "IndianRed") +
  geom_line(alpha = 0.7, aes(col = -error), lwd = 1, show.legend = FALSE) + 
  labs(y = "Estimated Misclassification Rate", x = "Maximum Leaf Size", ttitle = "Cross Validation of Tree Model") +
  geom_vline(xintercept = minNode, col = "purple") +
#  geom_hline(yintercept = minval$error) +
  theme_classic()
\end{lstlisting}

The plot suggests that the minimum misclassification rate for this model
will occur when the nodesize is limited to
\passthrough{\lstinline!r minNode!}.

\hypertarget{prune-the-tree}{%
\paragraph{Prune the Tree}\label{prune-the-tree}}

Prune the tree by using the \passthrough{\lstinline!prune.misclass!}
function:

\begin{lstlisting}
Cars.Clas.prunce.cv <- prune.misclass(CatSalesModTree, best = minNode)
plot(Cars.Clas.prunce.cv)
text(Cars.Clas.prunce.cv, pretty = 0, cex = 0.7)
\end{lstlisting}

\hypertarget{asess-the-pruned-model-against-the-test-data}{%
\paragraph{Asess the pruned model against the test
data}\label{asess-the-pruned-model-against-the-test-data}}

Compare the model returned by CV to the unseen testing data:

\begin{lstlisting}
test.preds <- predict(object = Cars.Clas.prunce.cv, newdata = CSeat[-train, -1], type = "class")
test.obs   <- CSeat[-train, -1]$CatSales

cfConfMat <- confusionMatrix(data = test.preds, reference = test.obs)
mcr <- 1-cfConfMat$overall[1]
\end{lstlisting}

from this it can be determined that the misclassification rate for this
pruned model is now \passthrough{\lstinline!r percent(round(mcr, 2))!}

This tree represents the best trade off between a model with too much
variance and a model with too much bias.

\hypertarget{regression-trees}{%
\subsection{Regression Trees}\label{regression-trees}}

Consider the \passthrough{\lstinline!Boston!} Data set:

\begin{lstlisting}
# kable(head(Boston))
head(Boston)
# kabstr(Boston)
# dim(Boston)
# summary(Boston)



\end{lstlisting}

This data set is a measurement of the crime rate throughout the Boston
area.

\hypertarget{seperate-training-data}{%
\subsubsection{Seperate training Data}\label{seperate-training-data}}

Split the data into a training and test set in order to evaulate the
difference caused by the cross validation technique

\begin{lstlisting}
set.seed(1)
train = sample(1:nrow(Boston), size = 0.52*nrow(Boston))
names(Boston)
\end{lstlisting}

\hypertarget{fit-a-regression-tree-to-the-training-data}{%
\subsubsection{Fit a Regression Tree to the Training
Data}\label{fit-a-regression-tree-to-the-training-data}}

In order to fit the regression tree to the training data use the train
variable as a subset:

\begin{lstlisting}
bost.mod.tree <- tree(formula =medv ~., data = Boston, subset = train)
# summary(bost.mod.tree)
plot(bost.mod.tree)
text(bost.mod.tree, pretty = 0, cex = 0.7)

\end{lstlisting}

\hypertarget{evalueate-the-performance}{%
\paragraph{Evalueate the Performance}\label{evalueate-the-performance}}

\begin{lstlisting}
modSum <- summary(bost.mod.tree)
rss <- modSum$dev
rmse <- sqrt(modSum$dev/length(length(bost.mod.tree$y)))
rmsernd <- signif(rmse, 1) * 1000
leaves <- summary(bost.mod.tree)$size
\end{lstlisting}

This model has \passthrough{\lstinline!r leaves!} terminal nodes, the
performance of this model can be evaulated by measuring the
\textbf{\emph{RSSS}} which is evaulated by the mode, this provides that
the average distance from data point to model prediction is \(\pm\)
\passthrough{\lstinline!r signif(rmse, 1)!} K USD.

\hypertarget{prune-the-tree-with-cv}{%
\subsubsection{Prune the tree with CV}\label{prune-the-tree-with-cv}}

\hypertarget{perform-cross-validation}{%
\paragraph{Perform Cross Validation}\label{perform-cross-validation}}

\begin{lstlisting}
cvvals <- cv.tree(object = bost.mod.tree, K = 10)
error <- sqrt(cvvals$dev)/length(bost.mod.tree$y)
TermNodes <- factor(x = cvvals$size, ordered = TRUE)

dvdf <- tibble(TermNodes, error) 



#Return the best number of leaves
minerror <- min(dvdf$error)
optn <- min(dvdf$TermNodes[dvdf$error==minerror]) #There could be multiple optimum values
\end{lstlisting}

\hypertarget{visualise-the-validaiton-error}{%
\paragraph{Visualise the validaiton
Error}\label{visualise-the-validaiton-error}}

\begin{lstlisting}
ggplot(dvdf, aes(x = TermNodes, y = error, group = 1)) +
  geom_point(col = "IndianRed", size = 3) +
  geom_line(col = "RoyalBlue") +
  geom_vline(xintercept = as.numeric(optn), col = "purple") +
  theme_classic() +
  labs(y = "Expected Error From model to Unseen Data ($50 K)", x = "Number of Terminal Nodes")
\end{lstlisting}

The model suggests that the appropriate number of terminal nodes to
choose is \passthrough{\lstinline!r optn!} and hence the model ought to
be left unchanged, however we will prune it for the sake of practice

\hypertarget{prune-the-tree-1}{%
\paragraph{Prune the Tree}\label{prune-the-tree-1}}

In order to prune the tree use the \passthrough{\lstinline!prune.tree!}
function and specify \passthrough{\lstinline!best=!} as the desired
number of leaves:

\begin{lstlisting}
prun.reg.mod <- prune.tree(tree = bost.mod.tree, best = 5)
plot(prun.reg.mod)
text(prun.reg.mod, pretty = 0, cex = 0.7)
\end{lstlisting}

\hypertarget{using-rpart}{%
\subsubsection{Using rpart}\label{using-rpart}}

This could have all been done in one call to rpart like so, however here
the data is not split in order to create a more accurate model.

\begin{lstlisting}
BosnonSyn <- Boston
bosTree <- rpart(formula = medv ~ ., data = Boston)
rpart.plot(bosTree)

nicenames <- c("Crime Rate", "Large Zoning Proportion", "Proportion of Industry", "Near River?", "NO Polution Level PPM", "Number of Rooms", "Proportion of 'Old' Buildings", "Distance to CBD's", "Accessability of Highway", "Tax Rate", "Ratio of Teachers", "Number of Residents of African-Americans Descent", "Property Value")
\end{lstlisting}
 
\end{document}














